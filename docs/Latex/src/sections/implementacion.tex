\section{Implementación}
La implementación se desarrolla en etapas:

    1.- Estructuras de datos:

	\hspace{1cm}- Uso de grafos para modelar las relaciones entre usuarios.\\
	\hspace{1cm}- Implementación de listas enlazadas y tablas hash para gestionar datos asociados a cada usuario.

    2.- Algoritmos:

	\hspace{1cm}- Adaptación del algoritmo de Dijkstra para encontrar rutas más cortas.\\.
	\hspace{1cm}- Cálculo de la similitud de Jaccard para sugerencias basadas en intereses comunes.

    3.- Manejo de datos:

	\hspace{1cm}- Almacenamiento estructurado en directorios para cada usuario.\\
	\hspace{1cm}- Archivos separados para guardar seguidores, seguidos y datos generales de los usuarios.

    4.- Generadores dinámicos:

	\hspace{1cm}- Creación automática de usuarios y simulación de publicaciones.\\
	\hspace{1cm}- Generación de conexiones aleatorias y recomendaciones personalizadas. \\

    5.- Interfaz de interacción:
	\hspace{1cm}- Desarrollo de una interfaz de línea de comandos para manejar la creación, modificación y búsqueda de usuarios.\\
	\hspace{1cm}- Ejecución a través de parámetros para una interacción dinámica.

