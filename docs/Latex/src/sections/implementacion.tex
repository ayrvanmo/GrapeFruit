\section{Implementación}
La implementación se desarrolla en etapas:

    1.- Estructuras de datos:

	    - Uso de grafos para modelar las relaciones entre usuarios.
	    - Implementación de listas enlazadas y tablas hash para gestionar datos asociados a cada usuario.

    2.- Algoritmos:

	    - Adaptación del algoritmo PageRank a la red social para calcular la influencia de los usuarios.
	    - Cálculo de la similitud de Jaccard para sugerencias basadas en intereses comunes.

    3.- Manejo de datos:

	    - Almacenamiento estructurado en directorios para cada usuario.
	    - Archivos separados para guardar seguidores, seguidos y datos generales de los usuarios.

    4.- Generadores dinámicos:

	    - Creación automática de usuarios y simulación de publicaciones.
	    - Opcionalmente, integración con un generador de contenido basado en GPT para enriquecer los mensajes.

    5.- Interfaz de interacción:

	    - Implementación en Python para manejar la creación, modificación y búsqueda de usuarios.
	    - Ejecución a través de parámetros para una interacción dinámica.

