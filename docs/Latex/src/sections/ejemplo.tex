\section{Ejemplo de uso de la red social}
El sistema de red social permite a los usuarios interactuar y conectar de diversas maneras. A continuación, se presentan algunos ejemplos de cómo los usuarios pueden aprovechar las funcionalidades del sistema:

\textbf{Recomendación de conexiones}
\begin{itemize}
    \item El sistema internamente sugiere posibles conexiones basándose en intereses comunes entre usuarios, para simular el comportamiento de los usuarios en la web, utiliza un ``Lanzamiento de moneda'' con el fin de determinar el enlace de cada parte involucrada.
    \item Facilita la creación de nuevas relaciones y ampliando la red de contactos.
\end{itemize}

\textbf{Publicación de contenido}
\begin{itemize}
    \item Los usuarios pueden publicar mensajes y otros contenidos en sus perfiles.
    \item Las publicaciones se pueden visualizar desde el menu interactivo de uso, al revisar amigos o buscar un perfil especifico.
\end{itemize}

\textbf{Búsqueda Optimizada}
\begin{itemize}
    \item El sistema permite a los usuarios realizar búsquedas eficientes para encontrar perfiles y contenidos específicos.
\end{itemize}

\textbf{Visualización de intereses}
\begin{itemize}
    \item El sistema proporciona herramientas para visualizar los intereses de los usuarios, mostrando sus preferencias junto a su perfil, sin embargo se decidio por no mostrar el rango de similitud explicitamente.
    \item Ayuda a identificar conexiones potenciales y a comprender mejor las preferencias de la comunidad.
\end{itemize}
