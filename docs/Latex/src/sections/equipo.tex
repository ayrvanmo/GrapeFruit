\section{Gestión del equipo de trabajo}
El equipo de trabajo está compuesto por los siguientes integrantes: Ayrton Morrison, Emmanuel Velásquez, Manuel González y Pablo Gómez. La división de responsabilidades se organizó en dos grupos principales: el equipo de programación y el equipo de documentación.

\textbf{Equipo de programación}

El equipo de programación es responsable del diseño y desarrollo del código del sistema. Sus principales tareas incluyen:

\begin{itemize}
    \item \textbf{Diseño de la arquitectura del sistema:} Planificación de la estructura y componentes del sistema para asegurar una implementación eficiente y escalable.
    \item \textbf{Desarrollo del algoritmo de similitud de Jaccard:} Implementación del algoritmo para calcular las similitudes entre los intereses de los usuarios, optimizando la precisión de las recomendaciones.
    \item \textbf{Integración de módulos y pruebas:} Asegurarse de que todos los componentes del sistema funcionen de manera coherente y realizar pruebas para detectar y corregir errores.
    \item 
\end{itemize}

\textbf{Equipo de documentación}

El equipo de documentación se encarga de la redacción del informe y la elaboración de diagramas y documentación adicional. Sus principales responsabilidades son:

\begin{itemize}
    \item \textbf{Redacción del informe del proyecto:} Documentación detallada del desarrollo del sistema, incluyendo objetivos, metodologías y resultados.
    \item \textbf{Elaboración de diagramas:} Creación de diagramas de arquitectura del sistema y otros gráficos que faciliten la comprensión del proyecto.
    \item \textbf{Documentación técnica:} Desarrollo de manuales de usuario y guías técnicas para asegurar que otros desarrolladores puedan entender y trabajar con el sistema.
    \item \textbf{Revisión y edición:}Asegurarse de que toda la documentación sea clara, precisa y esté libre de errores.
\end{itemize}

\textbf{Coordinación y comunicación}

Para asegurar una colaboración efectiva entre ambos equipos, se establecieron las siguientes prácticas de gestión:

\begin{itemize}
    \item \textbf{Reuniones semanales:} Sesiones de actualización para revisar el progreso, discutir desafíos y planificar las siguientes etapas del proyecto.
    \item \textbf{Revisión de pares:} Proceso donde los miembros del equipo revisan y proporcionan retroalimentación sobre el trabajo de sus compañeros para mejorar la calidad del proyecto.
\end{itemize}

Estas prácticas garantizan que todos los miembros del equipo estén alineados con los objetivos del proyecto y que se mantenga una alta calidad en el desarrollo y la documentación.
