\section{Conclusiones}
El proyecto ’GrapeLoot' resultó ser una sólida aplicación de las estructuras de datos y los algoritmos en un entorno simulado de redes sociales.  La implementación de estructuras de datos como grafos, tablas hash y listas enlazadas y algoritmos avanzados como el PageRank y la similitud de Jaccard, permitieron abordar problemas reales de gestión de datos y personalización de recomendaciones de forma bastante dinámica y eficiente, lo cual es un punto a favor en la implementación de la red social simulada.

En adicion se consiguio implementar un sistema de manejo de archivos eficiente, dinamico y con un control de memoria destacable, lo cual indica nuevamente que se logro implementar estructuras de datos eficientes, las cuales en combinacion con un manejo de archivos y directorios cauteloso, logro hacer un sistema eficiente de creacion dinamica y automatica de usuarios mientras se utiliza el programa. Simulando el uso de una red social real.


En resumen, este proyecto fue bastante exitoso, y si bien no fue fácil de implementar, debido a que se requiere mucha capacidad de abstracción e incluso también cierto dominio matemático para la parte de los algoritmos, fue una excelente forma de aplicar y consolidar los conocimientos adquiridos sobre estructuras de datos y algoritmos, lo que también servirá para irse adecuando a problemas de mayor complejidad a futuro.

